ESIPUHE VEKTOREISTA

\newpage
\subsection*{Matematikan opetussuunnitelma}

Lukion pitkän matematiikan kurssien tavoitteena on, että opiskelija ymmärtää matemaattisen ajattelun perusteet ja oppii ilmaisemaan itseään matematiikan kielellä sekä mallintamaan itse käytännön asioita matemaattisesti. Matematiikan pitkä oppimäärä antaa hyvät valmiudet luonnontieteiden opiskeluun.

Uusimpien (vuoden 2003) lukion opetussuunnitelman perusteiden mukaan pitkän matematiikan ensimmäisen kurssin tavoitteena on, että opiskelija
%\luettelo{
%§ vahvistaa yhtälön ratkaisemisen ja prosenttilaskennan taitojaan
%§ syventää verrannollisuuden, neliöjuuren ja potenssin käsitteiden ymmärtämistään
%§ tottuu käyttämään neliöjuuren ja potenssin laskusääntöjä
%§ syventää funktiokäsitteen ymmärtämistään tutkimalla potenssi- ja eksponenttifunktioita
%§ oppii ratkaisemaan potenssiyhtälöitä.
%}

Opetussuunnitelman perusteet määrittelevät kurssin keskeisiksi sisällöiksi
%\luettelo{
%§ potenssifunktion
%§ potenssiyhtälön ratkaisemisen
%§ juuret ja murtopotenssin
%§ eksponenttifunktion.
%}

\laatikko[Avoimet oppimateriaalit ry]{
Yhdistys tuottaa ja julkaisee oppimateriaaleja ja kirjoja, jotka ovat kaikille ilmaisia ja vapaita käyttää, levittää ja muokata. Vapaa matikka -sarja on suunnattu lukion pitkän matematiikan kursseille ja täyttää valtakunnallisen opetussuunnitelman vaatimukset.}

\newpage

\subsection*{Kirjan käyttämisestä}

Jos kirja ei anna tarpeeksi esimerkkejä jostakin asiasta, \textbf{ota yhteyttä}, ja asia korjataan välittömästi. \textbf{Oletusarvoisesti ymmärtämättömyyden syy ei ole sinun vaan oppimateriaalin.} Ota toki myös yhteyttä, jos sinulla on mielenkiintoisia tehtäväideoita!

Kirjan tehtävät on suurimmassa osaa luvuista jaettu kolmeen kategoriaan: Perusteet, kokonaisuuden hallitseminen ja lisätehtävät. Perustehtävät tarkistavat luvun määritelmien ja keskeisten tulosten ymmärtämisen sekä perustavanlaatuiset taidot -- nämä täytyy osata kurssin läpäisemiseen. Hallitse kokonaisuus -osiossa tehtävät ovat soveltavampia, ja ne yhdistelevät monipuolisesti luvun eri menetelmiä ja myös aiemmissa luvuissa opittuja asioita. Jos kaipaat lisäharjoitusta, sekä perustehtäviä että sekalaisia sovelluksia löytyy lisää kolmannesta osiosta. Yhteiskunnallisia tilanteita soveltavien tehtävien tiedot ovat faktuaalisia, ellei toisin mainita. Lisää tehtäviä -osion tähdellä $\star$ merkityt tehtävät ovat selvästi opetussuunnitelman ulkopuolisia tehtäviä, erittäin soveltavia, oppiainerajat ylittäviä tehtäviä, pitkiä tai tekijöiden mielestä muuten koettu mielenkiintoisiksi tai haastaviksi. Näiden tehtävien osaaminen ei missään nimessä ole vaatimus kympin kurssiarvosanaan.

Kirjan lopusta löytyy ylimääräinen teorialiite lukujärjestelmistä, kertaustehtäviä, harjoituskokeita, vanhoja yo- ja eri alojen valintakokeita, harjoitustehtävien vastaukset sekä asiasanahakemisto.

\newpage