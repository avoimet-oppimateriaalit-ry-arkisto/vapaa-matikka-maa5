\subsection{perusteita}


Vektoreilla on suunta ja pituus. Niitä on tapana kuvata nuolilla, joissa nuolen kärki ilmaisee vektorin suunnan. Vektoria ei kuitenkaan ole sidottu tiettyyn pisteeseen, vaan kaksi nuolta joilla on sama pituus ja suunta kuvaavat samaa vektoria.

(Kuva, jossa kaksi samaa vektoria esittävää nuolta.)

Vektoreita on tapana merkitä pienellä kirjaimella, jonka päällä on viiva. Vektorin $\bar{a}$ pituudelle puolestaan käytetään merkintää $|\bar{a}|$.

Toinen usein käytetty tapa on nimetä ensin vektoria esittävän nuolen alkupiste ja loppupiste isoilla kirjaimilla, jonka jälkeen vektorille käytetään merkintää $\overline{AB}$, missä $A$ on sen alkupiste ja $B$ sen loppupiste. Tämä tapa on hyödyllinen jos vektorien avulla halutaan piirtää monikulmioita, joiden kärkipisteet halutaan nimetä.

(Kuva monikulmiosta vektoreilla piirrettynä.)

Vektorien suuntia voidaan vertailla. Kaksi vektoria ovat yhdensuuntaisia jos niiden kautta kulkevat suorat ovat yhden suuntaisia. Muulloin vektorit ovat eri suuntaisia. Vektorien $\bar{a}$ ja $\bar{b}$ yhdensuutaisuudelle käytetään merkintää $\bar{a} \parallel \bar{b}$ ja erisuuntaisuudelle $\bar{a} \nparallel \bar{b}$.

(Kuva kolmesta vektorista ja niiden kautta kulkevista suorista, niin että kaksi vektoreista on yhdensuuntasia ja yksi erisuunantainen + kuvatekstiin selitys mitkä mitäkin.)

Yhdensuuntaiset vektorit voivat olla joko samansuuntaisia tai vastakkaissuuntaisia. Samansuuntaisuudelle käytetään merkintää $\bar{a} \uparrow \uparrow  \bar{b}$ ja vastakkaissuuntaisuudelle $\bar{a} \uparrow \downarrow \bar{b}$.

(Kuva kolmesta vektorista, jotka ovat kaikki eri pitusia, mutta yhdensuuntaisia, niin että kaksi vektoreista on samansuuntasia ja yksi niiden kanssa vastakkaissuunantainen + kuvatekstiin selitys mitkä mitäkin.)

Jos vektorit $\bar{a}$ ja $\bar{b}$ ovat sekä samansuutaisia ($\bar{a} \uparrow \uparrow  \bar{b}$) että samanpitusia ($|\bar{a}| = |\bar{b}|$), ovat vektorit samoja ($\bar{a} = \bar{b}$).

Vektorin $\bar{a}$ vastavektori $-\bar{a}$ on vektori, joka on sen kanssa vastakkaisuuntainen ($\bar{a} \uparrow \downarrow -\bar{a}$) ja samanpituinen ($|\bar{a}| = |-\bar{a}|$).

(Kuva vektorista ja sen vastavektorista.)

Vektoria, jonka pituus on nolla sanotaan nollavektoriksi. Nollavektorilla ei ole suuntaa, koska sen alkupiste ja loppupiste ovat samoja. Toisin sanoen jokainen muotoa $\bar{AA}$ oleva vektori on nollavektori.
 