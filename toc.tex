%\Opensolutionfile{ans}[content/vastaukset]
%\part{Vektorit}
%\input{content/vektorit}
%\part{Matriisit}
%\input{content/matriisit}
%\Closesolutionfile{ans}
%
%\section{Vastaukset}
%\input{content/vastaukset}

\providecommand{\lukufilter}[2]{#2} % ylikirjoitetaan kaanna_luku.sh -skriptistä.
\newcommand{\osa}[1]{\chapter{#1}} % osa
\newcommand{\nosa}[1]{\chapter*{#1} \addcontentsline{toc}{chapter}{#1}} % numeroimaton osa
\newcommand{\luku}[2]{\section{#2} \lukufilter{#1}{\input{content/TEORIA_#1} \input{content/TEHT_#1}}} % luku
\newcommand{\nluku}[2]{\section*{#2} \addcontentsline{toc}{section}{#2} \lukufilter{#1}{\input{content/#1}}} % numeroimaton luku
\newcommand{\vast}{\section*{Vastaukset} \addcontentsline{toc}{section}{Vastaukset} \begin{vastaussivu} \begin{Vastaus}{1}
	pöö
	
\end{Vastaus}
\begin{Vastaus}{2}
	pöö2
	
\end{Vastaus}
 \end{vastaussivu}}

\Opensolutionfile{ans}[content/LIITE_vastaukset] % kirjoittaa vastaukset tiedostoon content/LIITE_vastaukset.tex

\newpage
\nluku{LIITE_esipuhe}{Esipuhe}

\osa{Vektorit tasossa}
    \luku{vektorit}{Johdatus vektoreihin}
     \luku{vektoriesitys}{Vektorien esitystavat}
%    \luku{vektorisumma}{Vektorien summa ja erotus}
%    \luku{vektorien kertominen luvulla}{Vektorien kertominen luvulla}
%    \luku{vektorien kertominen luvulla}{Vektorien kertominen luvulla}
%    \luku{kanta}{Kantavektorit}
%       
%\osa{Vektorien koordinaattiesitys}
%    \luku{}{}    
%    \luku{pistetulo}{Pistetulo}
%
%\osa{Vektorit avaruudessa}
%    \luku{}{}
%    
%\osa{Suora ja taso avaruudessa}
%    \luku{}{}    
%    
%\osa{Ylikurssiasia}
%    \luku{}{}    
    
    
    
    

\Closesolutionfile{ans}
\vast
%\liitetyyli